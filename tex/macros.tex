% This file is partially based on macros.tex of Dr. Jakob Verbeek.
% Gokberk Cinbis, 2021

\usepackage{color}
\usepackage{xcolor}
\usepackage{soul}
\usepackage[normalem]{ulem}

% in-text to-do note macros
\def\todo#1{{\color{blue}{\hl{[#1]}}}}
\def\addcite{{\color{blue}{\hl{[cite]}} }} % no-arg citation-needed remark, usage: \addcite
\def\tocite#1{{\color{blue}{\hl{[cite: #1]}}}} % \tocite{comments}
\def\gc#1{{\color{red}{#1}}} % gokberk cinbis

\def\mypar#1{\vspace{0.15cm}\noindent{\bf #1.}}

% create a file 'darkmode.tex' to invert background/text color, for vampire-friendly rendering.
% it is a good idea to git-ignore darkmode.tex, not to interfere with co-author(s)' pdf appearances.
% optionally override default myfg, mybg definitions within that file.
\IfFileExists{darkmode.tex}{
    \usepackage{pagecolor}
    \definecolor{myfg}{gray}{0.94} 
    \definecolor{mybg}{gray}{0}
    \input{darkmode.tex} % myfg and mybg can be altered in darkmode.tex
    \pagecolor{mybg}
    \color{myfg}
}{} 

\usepackage{amsmath,amssymb,amsbsy,xspace}
\def\Eq#1{Eq.~(\ref{eq:#1})}
\def\R{{\rm I\!R}}                            	% the real numbers
\def\grad#1#2{\frac{\partial #1}{\partial #2}}  % gradient
\def\Ind#1{[\![#1]\!]}				% indicator function
\def\Tr#1{\textrm{Tr}\left[#1\right]}                % trace
\def\diag#1{\textrm{diag}\left( #1 \right)} % diag
\def\Gauss#1#2#3{\mathcal{N}\left(#1 ; #2,#3\right)}   % Gaussian density
\def\half{\frac{1}{2}}              % half
\def\<{\langle}
\def\>{\rangle}

% expectation: https://tex.stackexchange.com/questions/56765/getting-the-expectation-symbol-to-behave-like-sum-instead-of-sigma
\DeclareMathOperator*{\ExpectSign}{\mathbb{E}}       % expectation sign
%\def\ex#1#2{\textrm{I\!E}_{#1}\!\left[#2\right]}   % expectation (old)
\def\ex#1#2{\ExpectSign_{#1}\left[#2\right]}        % expectation (new)

\makeatletter
\DeclareRobustCommand\onedot{\futurelet\@let@token\@onedot}
\def\@onedot{\ifx\@let@token.\else.\null\fi\xspace}

\def\eg{\emph{e.g}\onedot} \def\Eg{\emph{E.g}\onedot}
\def\ie{\emph{i.e}\onedot} \def\Ie{\emph{I.e}\onedot}
\def\cf{\emph{c.f}\onedot} \def\Cf{\emph{C.f}\onedot}
\def\etc{\emph{etc}\onedot} \def\vs{\emph{vs}\onedot} 
\def\wrt{w.r.t\onedot} \def\dof{d.o.f\onedot}
\def\etal{\emph{et al}\onedot}

\makeatother

